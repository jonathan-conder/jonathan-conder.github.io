\documentclass[10pt]{article}

\usepackage[hidelinks]{hyperref}
\usepackage{enumitem}
\usepackage{fontawesome5}
\usepackage[a4paper, top = 15ex, vmarginratio = 2:3]{geometry}
\usepackage{microtype}
\usepackage[sfdefault]{noto}
\usepackage{stringstrings}
\usepackage{titlesec}
\usepackage{titling}
\usepackage{xcolor}

\pagestyle{empty}
\setlength\parindent{0em}
\renewcommand\baselinestretch{1.125}
\setlist{labelwidth=0.25em, leftmargin=!}
\SetLabelAlign{middle}{\hfill{}#1\hfill}
\newlist{iconize}{itemize}{1}
\setlist[iconize]{align=middle, labelwidth=1.25em, noitemsep}

\newcommand\asidefill[1]{{#1}\hspace{1em}\leavevmode\leaders\hrule height 0.75ex depth -0.5ex\hfill\kern 0em}
\newcommand\asideformat{\bfseries\color{black!62.5}}
\newcommand\aside[1]{\hfill{\footnotesize\asideformat\raisebox{1pt}{#1}}}

\titleformat\section{\large\asideformat}{}{0em}{\asidefill}
\titlespacing\section{0em}{1.5ex}{1ex}

\title{Resume}
\author{Jonathan Conder}
\newcommand\theemail{jono.conder@gmail.com}
\newcommand\thenumber{+64 20 4095 4837}
\newcommand\thegithub{github.com/jonathan-conder}
\newcommand\thegithubsm{github.com/jonathan-conder-sm}
\newcommand\thelinkedin{linkedin.com/in/jonathan-conder}

\convertchar[e]{\thenumber}{ }{}
\convertchar[e]{\thestring}{(}{} % chktex 9
\convertchar[e]{\thestring}{)}{} % chktex 9
\convertchar[q]{\thestring}{-}{}
\edef\thenumberplain{\thestring}

\hypersetup{pdftitle = {\thetitle}, pdfauthor = {\theauthor}, final}

\begin{document}

\begin{minipage}[t]{0.5\textwidth}
	\begin{iconize}
	\item[\faUser]
		\theauthor
	\item[\faEnvelope]
		\href{mailto:\theemail}{\theemail}
	\item[\faPhone]
		\href{tel:\thenumberplain}{\thenumber}
	\end{iconize}
\end{minipage}
\begin{minipage}[t]{0.5\textwidth}
	\begin{iconize}
	\item[\faGithub]
		\href{https://\thegithub}{\thegithub}
	\item[\faGithub]
		\href{https://\thegithubsm}{\thegithubsm}
	\item[\faLinkedin]
		\href{https://www.\thelinkedin}{\thelinkedin}
	\end{iconize}
\end{minipage}

\section{Profile}

Software engineer with expertise in backend development, DevOps, machine learning and systems programming.
Proficient Android/frontend developer.
PhD in Mathematics and BSc in Computer Science.

\section{Employment}

\begin{itemize}
\item
	Senior R\&D Software Engineer at Soul Machines. \aside{2020--2024}

	Part of a small team of researchers and engineers that developed AI models which drove the behaviour of realistic virtual characters.
	Responsible for two Python services which hosted some of these models, using FastAPI and PyTorch.
	Contributions:
	\begin{itemize}
	\item
		Trained and evaluated models, typically transformers or convolutional neural networks, to perform tasks such as language modeling and object detection.
	\item
		Created and maintained REST APIs which exposed these models to other services.
	\item
		Configured cloud deployments on AWS and Azure using Kubernetes and Terraform.
		Submitted bug reports and patches to Amazon EKS-D and related projects.
	\item
		Simplified deployment with Poetry and Docker.
		Used BuildKit and other techniques to write cache-friendly Dockerfiles.
		This made typical image uploads over 100$\times$ faster.
	\end{itemize}

	Contributed to a large cross-platform codebase, written in C, C++ and Julia.
	Highlights:
	\begin{itemize}
	\item
		Created a high-level library for running TensorFlow Lite models, using C++ and OpenCV.
		It uses data-oriented design extensively, resulting in it running 4$\times$ faster than Google's equivalent, with 100$\times$ less variance in processing time.
	\item
		Maintained a Julia fork.
		Contributed patches upstream whenever possible.
	\item
		Rewrote our C/Julia interface to improve overall performance by 10-15$\%$.
	\end{itemize}

	Improved developer experience by taking the initiative to:
	\begin{itemize}
	\item
		Review pull requests and assist with debugging regressions.
	\item
		Write GitHub Actions in JavaScript to implement CI/CD for Julia and Python code.
	\item
		Simplify and automate builds using CMake imported targets and Conan packages.
	\item
		Implement remote debugging for Julia by creating a VS Code extension in TypeScript.
	\item
		Document code using Documenter.jl, Doxygen and Sphinx.
	\end{itemize}
\end{itemize}

\section{Skills}

\begin{itemize}
\item
	Quickly understanding and solving problems, in both familiar and new environments.
\item
	Communicating clearly and resolving misunderstandings.
\item
	Working effectively in both independent and team settings.
\item
	Assisting others when needed and helping them grow professionally.
\end{itemize}

\section{Education}

\begin{itemize}
\item
	PhD in Mathematics (Algebraic Geometry), University of California San Diego. \aside{2013--2019}
\item
	BSc (Honours) in Mathematics, University of Auckland. \aside{2012\phantom{--2012}}
\item
	Conjoint BSc in Computer Science/BA in Mathematics, University of Auckland. \aside{2009--2011}
\end{itemize}

\end{document}
